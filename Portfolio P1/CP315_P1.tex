\documentclass[12pt]{article}
\usepackage{hyperref}
\usepackage{amsmath}

\begin{document}
\title{CP315 - Portfolio Part 1}
\date{February 26, 2017}
\author{Mason Cooper\\ Wilfrid Laurier University}
\maketitle
\clearpage
\tableofcontents
\clearpage

\section{Horner's Scheme}
The following is my code for Horner's scheme:\\

$horner := proc (f, x)$\\
$local d, h; h := f;$\\
$d := degree(h);$\\
$numapprox[hornerform](select(u->degree(u, x) <= d end proc, h), x)$\\
$end proc$\\
\begin{itemize}
\item $x^8-x^7+x^4-x^3+x+1$
$$1+(1+(-1+(1+(-1+x)x^3)x)8x^2)x$$
\item $x^{400}-x^{300}+x^{200}-x^{100}+1$
$$1+(-1+(1+(x^1{00}-1)x^{100})x^{100})x^{100}$$
\item $x^14+x^11+x^8+x^5$
$$(1+(1+(x^3+1)x^3)x^3)x^5$$
\end{itemize}
\clearpage
\section{Taylor's theorem}
My Taylor's theorem code,\\

$mytaylor := proc (f, x0, d) local r, i;$\\
$r := eval(f, x = x0);$\\
$for i to d  do  r := r+(eval(diff(f, xi),$\\
$x = x0))*(x-x0)^i/factorial(i)$\\
$end do$\\
$end proc$\\

The following is the output from my code for Taylor's theorem and the output from Maple's built in Taylor command for comparison:\\

\begin{itemize}
\item $f(x)=cos(x)$, up to degree $k=10$, around $x_0=0$
$$1-\frac{1}{2}x^2+\frac{1}{24}x^4-\frac{1}{720}x^6+\frac{1}{40320}x^8-\frac{1}{3628800}x^{10}$$

$$1-\frac{1}{2}x^2+\frac{1}{24}x^4-\frac{1}{720}x^6+\frac{1}{40320}x^8+O(x^{10})$$\\

\item $f(x)=e^{x}$, up to degree $k=10$, around $x_0=0$
$$1+ln(e)x+\frac{1}{2}ln(e)^2x^2+\frac{1}{6}ln(e)^3x^3+\frac{1}{24}ln(e)^4x^4+\frac{1}{120}ln(e)^5x^5\\$$
$$+\frac{1}{720}ln(e)^6x^6+\frac{1}{5040}ln(e)^7x^7+\frac{1}{40320}ln(e)^8x^8+\frac{1}{3628800}ln(e)^9x^9+\frac{1}{3628800}ln(e)^{10}x^{10}$$

$$1+ln(e)x+\frac{1}{2}ln(e)^2x^2+\frac{1}{6}ln(e)^3x^3+\frac{1}{24}ln(e)^4x^4+\frac{1}{120}ln(e)^5x^5\\$$
$$+\frac{1}{720}ln(e)^6x^6+\frac{1}{5040}ln(e)^7x^7+\frac{1}{40320}ln(e)^8x^8+\frac{1}{3628800}ln(e)^9x^9+O(x^{10})$$
\end{itemize}
\section{Polynomials}
My code for sign change,\\

$signchange := proc (f)$\\
$local max, min, i, h, result, t, s; h := f;$\\
$result := [];$\\
$max := 100;$\\
$min := -100;$\\
$t := eval(f, x = min);$\\
$for i from min to max do$\\
$s := eval(f, x = i);$\\
$if s = 0 then$\\
$result := [op(result), [i]];$\\
$s := eval(f, x = i+1);$\\
$if 0 < t and 0 < s or t < 0 and s < 0 then$\\
$result := [op(result), [i]]$\\
$else$\\
$t := s$\\
$end if$\\
$elif 0 < t and s < 0 then$\\
$result := [op(result), [i-1, i]];$\\
$t := s$\\
$elif t < 0 and 0 < s then$\\
$result := [op(result), [i-1, i]];$\\
$t := s$\\
$end if$\\
$end do;$\\
$return result$\\
$end proc$\\
\clearpage
My Vieta formulae,\\

$vieta := proc (f, i)$\\
$local j, d, a, h, s;$\\
$h := f;$\\
$a := [];$\\
$d := degree(f);$\\
$for j from 0 to d do$\\
$a := [op(a), coeff(h, x, j)]$\\
$end do;$\\
$d := d+1;$\\
$s := (-1)^i*\frac{a[d-i]}{a[d]};$\\
$return s$\\
$end proc$\\

With my sign change code, the function $x^3-44x^2+564x-1728$ had sign changes at,
$$[[4, 5], [18], [21, 22]]$$
Using Vieta's formulae I was able to compute the following,
$$p_1+p_2+p_3 = 44$$
$$p_1p_2p_3 = 1728$$
$$p_1p_2+p_1p_3 +p_2p_3 = 564$$
$$p_1^2+p_2^2+p_3^2 = 808$$
$$p_1^3+p_2^3+p_3^3 = 15920$$
$$\frac{1}{p_1^2}+\frac{1}{p_2^2}+\frac{1}{p_3^2} = 808$$
\clearpage
\section{Equation Solving}
\subsection{Bisection method}
My code for bisection method,\\

$bisection := proc (l, h, f)$\\
$local a, b, c, m, n;$\\
$Digits := 4;$\\
$a := l;$\\
$b := h;$\\
$while \frac{1}{10000} < (1/2)*b-(1/2)*a do$\\
$c := \frac{a+b}{2};$\\
$m := evalf(eval(f, x = c));$\\
$n := evalf(eval(f, x = a));$\\
$if m = 0 then$\\
$return r = c$\\
$end if;$\\
$if n*m < 0 then b := c$\\
$else$\\
$a := c$\\
$end if$\\
$end do;$\\
$return r = evalf(\frac{a+b}{2})$\\
$end proc$\\

The following is output of my code for bisection method,
\begin{itemize}
\item $f(x) = x^3+x-1, x \in [0,1]$
$$r = .6823$$
\item $f(x) = cosx - x, x \in [0,1]$
$$r = .7391$$\\
\end{itemize}
\subsection{Fixed-point iteration method}
My code for Fixed-point iteration method,\\

$FPI := proc (f, x0, max)$\\
$local p0, p1, i;$\\
$Digits := 9;$\\
$p0 := evalf(x0);$\\
$p1 := p0;$\\
$for i from 0 to max do$\\
$p0 := p1;$\\
$p1 := eval(f, x = p0);$\\
$i := i+1$\\
$end do;$\\
$return p1$\\
$end proc$\\

The following is output of my code for Fixed-point iteration method,
\begin{itemize}
\item $g(x) = \frac{(x+\frac{2}{x})}{2}$
$$1.41421356$$
\item $x^3 = 2*x+2$
$$1.76931316$$
\end{itemize}
\clearpage
\subsection{Newton's method}
My code for Newton's method,\\

$newton := proc (f, x0, max)$\\
$local p0, p1, i;$\\
$Digits := 9;$\\
$h := f;$\\
$i := 0;$\\
$p0 := evalf(x0);$\\
$p1 := p0;$\\
$while i < max do$\\
$p0 := p1;$\\
$p1 := p0-(\frac{eval(h, x = p0))}{(eval((h', x), x = p0))};$\\
$i := i+1$\\
$end do;$\\
$return p1$\\
$end proc$\\

The following is output of my code for Newton's method,
\begin{itemize}
\item $f(x) = x^3+x-1, x_0 = 0.1$
$$0.682327804$$
\item $x^2+1 = 0, x_0 = 2]$
$$2.40087675$$
\end{itemize}
\clearpage
\section{Systems of Linear Equations}
\subsection{Gauss Elimination}
My code for Gauss Elimination method,\\

$GaussElimination := proc (A)$\\
$local B, X, i, j, k, d, c;$\\
$B := A;$\\
$d := Dimension(B);$\\
$X := [seq(x[i], i = 1 .. d[1])];$\\
$for j to d[1] do for i to d[1] do$\\
$if j < i then$\\
$c := \frac{B[i, j]}{B[j, j]};$\\
$for k to d[2] do$\\
$B[i, k] := B[i, k]-c*B[j, k]$\\
$end do$\\
$end if$\\
$end do$\\
$end do;$\\
$X[d[1]] := \frac{B[d[1], d[1]+1]}{B[d[1], d[1]]};$\\
$for i from d[1]-1 by -1 to 1 do$\\
$c := 0;$\\
$for j from i+1 to d[1] do$\\
$c := c+B[i, j]*X[j]$\\
$end do;$\\
$X[i] := \frac{B[i, d[2]]-c}{B[i, i]}$\\
$end do;$\\
$return X$\\
$end proc$\\

The following is output of my code for Gauss Elimination,
$$[1, 2, 4]$$
\subsection{Gauss-Jordan Elimination}
My code for Gauss-Jordan Elimination method,\\

$GaussJordan := proc (A, n)$\\
$local i, j, B, k;$\\
$B := A;$\\
$for i to n do$\\
$for j from i+1 to n do$\\
$if abs(B[i, i]) < abs(B[j, i])$\\
$then B[i, j] = B[j, i]$\\
$end if$\\
$end do;$\\
$B[i] := \frac{B[i]}{B[i, i]};$\\
$for k to n do$\\
$if k <> i$\\
$then B[k] := -B[i]*B[k, i]+B[k]$\\
$end if$\\
$end do$\\
$end do;$\\
$return B$\\
$end proc$\\

The following is output of my code for Gauss-Jordan Elimination,
$$\begin{bmatrix}
1 & 0 & 0 & 1\\
0 & 1 & 0 & 2\\
0 & 0 & 1 & 4
\end{bmatrix}$$
\clearpage
\subsection{The Hilbert matrix}
The following is output of my code for Gauss  Elimination and Gauss-Jordan Elimination on the Hilbert matrix,\\

Gauss Elimination, $H_5 \cdot \vec{x} = \vec{1}$
$$Determinant is, \frac{1}{266716800000}$$
$$[5, -120, 630, -1120, 630]$$

Gauss-Jordan Elimination, $H_5 \cdot \vec{x} = \vec{1}$ and Determinant,
$$\begin{bmatrix}
1 & 0 & 0 & 0 & 0 & 5\\
0 & 1 & 0 & 0 & 0 & -120\\
0 & 0 & 1 & 0 & 0 & 630\\
0 & 0 & 0 & 1 & 0 & -1120\\
0 & 0 & 0 & 0 & 1 & 630\\
\end{bmatrix}$$

Gauss-Jordan Elimination, $H_5^{-1}$
$$\begin{bmatrix}
1 & 0 & 0 & 0 & 0 & 25 & -300 & 1050 & -1400 & 630\\
0 & 1 & 0 & 0 & 0 & -300 & 4800 & -18900 & 26880 & -12600\\
0 & 0 & 1 & 0 & 0 & 1050 & -18900 & 79380 & -117600 & 56700\\
0 & 0 & 0 & 1 & 0 & -1400 & 26880 & -117600 & 179200 & -88200\\
0 & 0 & 0 & 0 & 1 & 630 & -12600 & 56700 & -88200 & 44100\\
\end{bmatrix}$$

Condition Number,\\

1-Form\\
$Multiply(MatrixNorm(H, 1), MatrixNorm(MatrixInverse(H), 1));$\\
$$943656$$\\

Frobenius\\
$Multiply(MatrixNorm(H, Frobenius), MatrixNorm(MatrixInverse(H), Frobenius))$\\
$$\frac{1}{504}\sqrt(3700542505)\sqrt(15871330)$$
\end{document}